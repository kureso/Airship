\documentclass[reqno, a4paper]{amsart}
\newcommand\hmmax{0}
\newcommand\bmmax{0}
\usepackage{amsmath}
\usepackage{amssymb}
\usepackage{amsthm}

% PAGE DIMENSION
\usepackage[scale=0.9]{geometry}

% BIBLIOGRAPHY
\usepackage[authoryear]{natbib}
\usepackage{bibentry} % inline refereces

% ENCODING, LANGUAGE
\usepackage[czech]{babel}
\usepackage[utf8]{inputenc}

% GRAPHICS
\usepackage{graphicx}
\usepackage{wrapfig}
\usepackage{subfig}

% HYPERTEXT, SOURCE CODE SPECIALS
\usepackage[unicode]{hyperref}
\usepackage[active]{srcltx} % use TeX-souce-specials-mode

% SYMBOLS, FONTS
\usepackage{mathbbol}
\usepackage{bm} % sophisticated \boldsymbol
%\usepackage{stmaryrd}
\usepackage{MnSymbol} % \lsem, \rsem, tensor product :
\usepackage{gensymb}
\usepackage{eurosym}

% UNITS, TYPESETTING TENSORS
%\usepackage{units}
\usepackage{siunitx}
\usepackage{tensor}
\usepackage{accents}

% COMPACT LIST ENVIRONMENT
\usepackage{enumitem}

% LINE NUMBERS
\usepackage{lineno}

% TABLE OF CONTENTS IN TWO COLUMNS
% \usepackage[toc]{multitoc} % It seems that it does not work with amsart
% the workaround is the command
% \addtocontents{toc}{\protect\begin{multicols}{2}} % workaround for table of contents in two columns in amsart documentclass
% see below
\usepackage{multicol}

% TODO NOTES
\usepackage{todonotes}

% TABLES
\usepackage{booktabs}
\usepackage{dcolumn}
\usepackage{tabularx}


% SOURCE CODE LISTINGS
\usepackage{listings}
%\usepackage{minted}
\usepackage{xcolor}

% IMPORT CSV files
\usepackage{csvsimple}

% NUMBERS
%\usepackage{numprint}

\numberwithin{equation}{section}
\let\cite\citet
\newcommand*{\doi}[1]{\href{http://dx.doi.org/#1}{doi: #1}}

\input{macros}
\input{listings-style}
\newcommand{\navstart}{A} % subscript denoting start of the navigation (time, initial positions, ...)
\newcommand{\navend}{B} % subscript denoting the end of the navigation (time, initial positions, ...)

\newcommand{\navext}{\mathrm{ext}} % extreme value

\title[Matematické kyvadlo]{Navigace vzducholodi}

\author{Ondrej Kureš, Marek Mikloš, Ladislav Trnka}

%\thanks{}


\begin{document}

\begin{abstract}
V tomto textu se pokusíme odpovědět na otázku, kudy má letět vzducholoď ve stacionárním větrném poli, aby do cíle doletěla v nejkratším možném čase. Teoreticky odvozené závěry pak budeme aplikovat na jednoduché rychlostní pole větru, kde trajektorie z různých bodů a časy letu numericky dořešíme.
\end{abstract}

\maketitle

\addtocontents{toc}{\protect\begin{multicols}{2}} % workaround for table of contents in two columns in amsart documentclass
\tableofcontents

%\linenumbers


\section{Úvod}
\label{sec:Úvod}

Vzducholoď se pohybuje ve větrném poli $\vec{w}$ a má za cíl překonat vzdálenost z bodu A do bodu B. V tomto textu se budeme zabývat otázkou jak zvolit její trasu, aby dorazila do cíle v nejkratším možném čase. Točení kormidla vzducholodi budeme charakterizovat jejím směrem letu tedy funkcí $\beta (t)$. Můžeme se ptát, jak točit kormidlem tak, aby vzducholoď dorazila do cíle co nejdříve. 

Trajektorii vzducholodi budeme popisovat v kartézských souřadnicích a to v rovině $(x,y)$, zanedbáme popis výšky. Vzducholoď se v bezvětří pohybuje rychlostí $\vec{V}$. Pro zjednodušení výpočtů uvažujme konstantní rychlost $\vec{V}$, stacionární pole $\vec{w}$, cílový bod B jako počátek souřadnic (lze vždy zajistit vhodnou transformací). Dále zanedbáváme zpoždění reakce vzducholodě na stočení kormidla.

\begin{figure}[h!]
  \centering
  \includegraphics[width=15cm]{figures/airship.eps}
  \caption{Nastínění uvažované situace.}
  \label{Rplot1-2}
\end{figure}

Pro okamžitou rychlost vzducholodi platí:
\begin{equation} 
\label{rce1}
\begin{split}
\dd{x}{t} & = V \cos \beta(t) + u(x,y), \\
\dd{y}{t} & = V \sin \beta(t) + v(x,y),
\end{split}
\end{equation}

kde $
\vec{x}(t)=\transpose{
  \begin{bmatrix}
    x(t) &
    y(t)
  \end{bmatrix}
}
$ je hledaná trajektorie, $\beta \in \langle 0, 2 \pi)$ je směr letu a $
\vec{w}
=
\transpose{
  \begin{bmatrix}
    u &
    v
  \end{bmatrix}
}
$ je dané pole větru. Dále známe:
\begin{subequations}
  \label{eq:1}
  \begin{align}
    \label{eq:2}
    \vec{x}(t_{A})& = A, \\
    \label{eq:3}
    \vec{x}(t_{B})& = B = \transpose{
  \begin{bmatrix}
    0 &
    0
  \end{bmatrix}
},
  \end{align}
\end{subequations}
kde $t_{A}$ je čas startu vzducholodi a $t_{B}$ je čas příletu\footnote{Při příletu vzducholoď nebude mít nulovou rychlost.}. 

\section{Variační počet}
\label{sec:Variační počet}

Náš zájem se proto soustřeďuje na minimalizaci funkcionálu:
\begin{equation}
  \label{eq:8}
  I(\beta, t_{B}) =_{\bydefinition} \int_{t_{A}}^{t_{B}}\, \diff t = t_{B} - t_{A},
\end{equation}
při splnění soustavy rovnic $\eqref{rce1}$, které kompaktněji přepišme jako:
\begin{equation}
  \label{eq:7}
  \dd{\vec{x}}{t} = \vec{f}(\vec{x}, \beta).
\end{equation}
Chceme tedy minimalizovat cestovní čas a přípustné trajektorie musí splňovat $\eqref{eq:7}$. Při hledání extremály využijme koncept vázaných extrémů a Lagrangeových multiplikátorů $\vec{\lambda}$. Proto studujme funkcionál:
\begin{equation}
  \label{eq:9}
  J(\beta, t_{B}) =_{\bydefinition}
  \int_{t = t_{A}}^{t_{B}}
  \left(
    1
    -
    \vectordot{\greekvec{\lambda}}{\left(\dd{\vec{x}}{t} - \vec{f}(\vec{x}, \beta) \right)}
  \right)
  \, \diff t,
\end{equation}
kde funkce $\vec{\lambda}$ bude upřesněna později. Nyní hledejme G\^{a}teauxovu derivaci $J(\beta, t_{B})$:
\begin{equation}
\begin{split}
  \label{eq:13}
  \Diff J(\beta, t_{B}) [(\alpha, \tau)]
  & =
  _{\bydefinition}
  \left.
    \dd{}{\varepsilon}
    J(\beta_{\navext} + \varepsilon \alpha, t_{\navend, \navext} + \varepsilon \tau)
  \right|_{\varepsilon = 0}  \\
  & =
  _{\bydefinition}
  \left.
    \left[
      \dd{}{\varepsilon}
      \int_{t = t_{A}}^{t_{\navend, \navext} + \varepsilon \tau}
      \left(
        1
        -
        \vectordot{\greekvec{\lambda}}{\left(\dd{\vec{x}_{\varepsilon}}{t} - \vec{f}(\vec{x}_{\varepsilon}, \beta) \right)}
      \right)
      \, \diff t
    \right]
  \right|_{\varepsilon = 0}
  ,
\end{split}
\end{equation}
při variaci:
\begin{subequations}
  \label{eq:10}
  \begin{align}
    \label{eq:11}
    \beta &= \beta_{\navext} + \varepsilon \alpha, \\
    \label{eq:12}
    t_{\navend} &= t_{\navend, \navext} + \varepsilon \tau,
  \end{align}
\end{subequations}
kde $\vec{x}_{\epsilon}$ je korespondující trajektorie k $\beta$ a $t_{B}$. Přičemž stále platí:
\begin{subequations}
  \label{eq:010}
  \begin{align}
    \label{eq:020}
    \vec{x}_{\epsilon}(t_{A})& = \vec{x}_{\navext}(t_{A})=A, \\
    \label{eq:030}
    \vec{x}_{\epsilon}(t_{\navend, \navext} + \varepsilon \tau) & =\vec{x}_{\navext}(t_{\navend, \navext})= B = \vec{0}.
  \end{align}
\end{subequations}
Nejdříve upravme $\eqref{eq:13}$ pomocí integrace per partes na člen $\vectordot{\greekvec{\lambda}}{\dd{\vec{x}_{\varepsilon}}{t}}$, některé členy budou dle předchozího nulové a dostáváme:
\begin{multline}
  \label{eq:18}
  \Diff J(\beta, t_{B}) [(\alpha, \tau)]
  =
  \left.
    \left[
      \dd{}{\varepsilon}
      \int_{t = t_{\navstart}}^{t_{\navend, \navext}}
      \left(
        1
        +
        \vectordot{\dd{\greekvec{\lambda}}{t}}{\vec{x}_{\varepsilon}}
        +
        \vectordot{\greekvec{\lambda}}{\vec{f}(\vec{x}_{\varepsilon}, \beta)}
      \right)
      \, \diff t
    \right]
  \right|_{\varepsilon = 0}
  +
  \left.
    \left[
      \dd{}{\varepsilon}
      \int_{t = t_{\navend, \navext}}^{t_{\navend, \navext} + \varepsilon \tau}
      \left(
        1
        +
        \vectordot{\dd{\greekvec{\lambda}}{t}}{\vec{x}_{\varepsilon}}
        +
        \vectordot{\greekvec{\lambda}}{\vec{f}(\vec{x}_{\varepsilon}, \beta)}
      \right)
      \, \diff t
    \right]
  \right|_{\varepsilon = 0}
  \\
  =
    \left.
    \left[
      \dd{}{\varepsilon}
      \int_{t = t_{\navstart}}^{t_{\navend, \navext}}
      \left(
        1
        +
        \vectordot{\dd{\greekvec{\lambda}}{t}}{\vec{x}_{\varepsilon}}
        +
        \vectordot{\greekvec{\lambda}}{\vec{f}(\vec{x}_{\varepsilon}, \beta)}
      \right)
      \, \diff t
    \right]
  \right|_{\varepsilon = 0}
  +
\left.
    \left[
        1
        +
        \vectordot{\greekvec{\lambda}}{\vec{f}(\vec{x}_{\navext}, \beta_{\navext})}
    \right]
  \right|_{t = t_{\navend, \navext}}
  \tau.
\end{multline}
Podle \citep{prusa.tuma} použijeme geniální trik:
$\vec{x}_{\varepsilon} \approx \vec{x}_{\navext} + \epsilon \vec{y} + \cdots$, kde zanedbáme členy vyššího řádu a kde $\vec{y}$ je funkce času. Tím dále můžeme upravit první člen v poslední rovnosti $\eqref{eq:18}$:
\begin{align}
    \left.
    \left[
      \dd{}{\varepsilon}
      \int_{t = t_{\navstart}}^{t_{\navend, \navext}}
      \left(
        1
        +
        \vectordot{\dd{\greekvec{\lambda}}{t}}{\vec{x}_{\varepsilon}}
        +
        \vectordot{\greekvec{\lambda}}{\vec{f}(\vec{x}_{\varepsilon}, \beta)}
      \right)
      \, \diff t
    \right]
  \right|_{\varepsilon = 0} =  \int_{t = t_{\navstart}}^{t_{\navend, \navext}}
  \left(
    \vectordot{\left[ \dd{\greekvec{\lambda}}{t} + \transpose{\left. \pd{\vec{f}}{\vec{x}} \right|_{\vec{x} = \vec{x}_{\navext}, \beta = \beta_{\navext}}} \greekvec{\lambda} \right]}{\vec{y}}
    +
    \vectordot{\greekvec{\lambda}}
    {
      \left. \pd{\vec{f}}{\beta} \right|_{\vec{x} = \vec{x}_{\navext}, \beta = \beta_{\navext}} \alpha
    }
  \right)
  \, \diff t.
\end{align}
Nyní můžeme přistoupit k vybrání $\vec{\lambda}$ takové, aby bylo splněno:
\begin{align}
  \label{eq:24}
  \dd{\greekvec{\lambda}}{t} = - \transpose{\left. \pd{\vec{f}}{\vec{x}} \right|_{\vec{x} = \vec{x}_{\navext}, \beta = \beta_{\navext}}} \greekvec{\lambda}
  .
\end{align}
Po dosazení dostáváme výsledný vztah pro G\^{a}teuxovu derivaci:
\begin{equation}
  \label{eq:25}
  \Diff J(\beta, t_{B}) [(\alpha, \tau)]
  =
  \int_{t = t_{\navstart}}^{t_{\navend, \navext}}
  \vectordot{\greekvec{\lambda}}
  {
    \left. \pd{\vec{f}}{\beta} \right|_{\vec{x} = \vec{x}_{\navext}, \beta = \beta_{\navext}} \alpha
  }
  \, \diff t
  +
  \left.
    \left[
        1
        +
        \vectordot{\greekvec{\lambda}}{\vec{f}(\vec{x}_{\navext}, \beta_{\navext})}
    \right]
  \right|_{t = t_{\navend, \navext}}
  \tau =0
  ,
\end{equation}
což musí platit pro libovolné $\alpha$ a $\tau$. Tímto dostáváme:
\begin{subequations}
  \label{eq:rce2}
  \begin{align}
    \label{eq:31}
    \vectordot{\greekvec{\lambda}}
    {
    \pd{\vec{f}}{\beta}(\vec{x}_{\navext}, \beta_{\navext}) 
    }
    &=
      0, \\
    \label{eq:32}
    \left.
    \left[
    1
    +
    \vectordot{\greekvec{\lambda}}{\vec{f}(\vec{x}_{\navext}, \beta_{\navext})}
    \right]
    \right|_{t = t_{\navend, \navext}}
    &=
      0.
  \end{align}
\end{subequations} 
Pokusme se vypočítat časovou derivaci funkce $1 +  \vectordot{\greekvec{\lambda}}{\vec{f}(\vec{x}_{\navext}, \beta_{\navext})}$, kam dosadíme z rovnic $\eqref{eq:7},\eqref{eq:31}$ a $\eqref{eq:24}$:
\begin{align}
\dd{}{t}
  \left(
    1 +  \vectordot{\greekvec{\lambda}}{\vec{f}(\vec{x}_{\navext}, \beta_{\navext})}
  \right) = 0,
\end{align}
potom funkce $1 +  \vectordot{\greekvec{\lambda}}{\vec{f}(\vec{x}_{\navext}, \beta_{\navext})}$ se musí rovnat konstantě po celý časový interval letu, ale z rovnice $\eqref{eq:32}$ vyplývá, že je rovna nule pro $t \in (t_{A},t_{B})$. Získáváme systém lineárních algebraických rovnic pro $\vec{\lambda}$:
\begin{subequations}
  \label{eq:40}
  \begin{align}
    \label{eq:41}
    1
    +
    \vectordot{\greekvec{\lambda}}{\vec{f}(\vec{x}_{\navext}, \beta_{\navext})}
    &=
      0
      ,
    \\
    \label{eq:42}
    \vectordot{\greekvec{\lambda}}
    {
    \pd{\vec{f}}{\beta}(\vec{x}_{\navext}, \beta_{\navext}) 
    }
    &=
      0.
  \end{align}
\end{subequations}
Řešením této soustavy pro naší pravou stranu $\eqref{rce1}$ můžeme získat explicitní vzorec pro $\vec{\lambda}$:
\begin{equation}
  \label{eq:45}
  \begin{bmatrix}
    \lambda_x \\
    \lambda_y
  \end{bmatrix}
  =
  \frac{1}{V +  u(x_{\navext}, y_{\navext}) \cos \beta_{\navext} +  v(x_{\navext}, y_{\navext}) \sin \beta_{\navext}}
  \begin{bmatrix}
    \cos \beta_{\navext} \\
    \sin \beta_{\navext}
  \end{bmatrix}
  .
\end{equation}
Odkud lze odvodit Zermelova navigační rovnice\footnote{Ernst Zermelo - přednáška v Praze 1931} s použitím rovnice $\eqref{eq:24}$ a užitím skalárního součinu obou stran rovnice s vektorem $
  \transpose{
  \begin{bmatrix}
    -\sin \beta_{\navext} &
     \cos \beta_{\navext}
  \end{bmatrix}
  }
$. Na otázku jak se vyvíjí optimální směr letu vzducholodě $\beta_{\navext}$, nám právě odpovídá Zermelova navigační rovnice:
\begin{equation}
  \label{eq:50}
  \dd{\beta_\navext}{t}
  =
  \pd{v}{x}(x_{\navext}, y_{\navext})
  \sin^2 \beta_\navext
  +
  \left(
    \pd{u}{x}(x_{\navext}, y_{\navext})
    -
    \pd{v}{y}(x_{\navext}, y_{\navext})
  \right)
  \sin \beta_\navext
  \cos \beta_\navext
  -
  \pd{u}{y}(x_{\navext}, y_{\navext})
  \cos^2 \beta_\navext
  .
\end{equation}
Tímto jsme odvodili všechny evoluční rovnice našeho problému.

\section{Řešení}
\label{sec:Řešení}

Máme zadané rychlostní pole větru $\vec{w}$, známe počáteční bod A a koncový bod B letu:
\begin{align}
\vec{w}
& =
\begin{bmatrix}
    u(x,y) \\
    v(x,y)
  \end{bmatrix}
,\\
A
& =
  \begin{bmatrix}
    A_{1} \\
    A_{2} \end{bmatrix}
,\\
B
& =
  \begin{bmatrix}
    0 \\
    0 \end{bmatrix}
.
\end{align}
Optimální trajektorii $\vec{x}_{\navext}$ a nejkratší možný čas letu $t_{B}-t_{A}$ hledáme řešením diferenciálních rovnic:
\begin{subequations}
  \label{eq:51}
  \begin{align}
    \label{eq:52}
    \dd{x_{\navext}}{t} &= V \cos \beta_{\navext} + u(x_{\navext}, y_{\navext}),  \\
    \label{eq:53}
    \dd{y_{\navext}}{t} &= V \sin \beta_{\navext} + v(x_{\navext}, y_{\navext}),  \\
    \label{eq:54}
      \dd{\beta_\navext}{t}
                              &=
  \pd{v}{x}(x_{\navext}, y_{\navext})
  \sin^2 \beta_\navext
  +
  \left(
    \pd{u}{x}(x_{\navext}, y_{\navext})
    -
    \pd{v}{y}(x_{\navext}, y_{\navext})
  \right)
  \sin \beta_\navext
  \cos \beta_\navext
  -
  \pd{u}{y}(x_{\navext}, y_{\navext})
  \cos^2 \beta_\navext
    ,
  \end{align}
\end{subequations}
pro neznámé $\vec{x}_{\navext}=
  \transpose{
  \begin{bmatrix}
    x_{\navext} &
    y_{\navext}
  \end{bmatrix}
  }
$ a $\beta_{\navext}$ při splnění podmínek:
\begin{align}
  \vec{x}_{\navext}(t_{A}) & = A ,\\
  \label{rce3}
  \vec{x}_{\navext}(t_{\navend} ; \beta_{\navext, 0}) & = \vec{0},
\end{align}
kde $\beta_{\navext}(t_{A})=\beta_{\navext, 0}$ je počáteční natočení vzducholodě. Neboli abychom mohli řešit soustavu diferenciálních rovnic $\eqref{eq:51}$, potřebujeme navíc kromě znalosti $\eqref{eq:1}$ ještě přidat další počáteční podmínku $\beta_{\navext, 0}$, pro kterou platí pouze rovnost $\eqref{rce3}$.

Jak si ukážeme v další sekci, pro vhodně zadané rychlostní pole větru lze napsat soustavu rovnic mezi $\vec{x}_{\navext}(t_{A})$ a $ \beta_{\navext, 0}$.

\section{Jednoduché veterné pole}
\label{sec:JVP}
Najjednoduchší prípad veterného poľa, by sme mohli uvažovať bezvetrie. V takom prípade bude pole charakterizované funkciami: $u(x,y)=0$, $v(x,y)=0$. Potom:
\begin{align}
\dd{\beta_\navext}{t} = 0,
\end{align}
z toho:
\begin{align}
\beta_\navext =\beta_{\navext,0},
\end{align}
čo znamená, že vzducholoď bude natočená priamo na cieľ.

Uvažujme teraz prípad lineárnej závislosti veterného poľa na polohe. V tomto prípade bude pole charakterizované funkciami:
\begin{align}
u &= -\frac{V}{h}y ,\\
v &= 0.
\end{align}

\begin{figure}
\includegraphics[scale=0.7]{figures/Vekt.pole.pdf}
\caption{Rychlostní pole větru.}
\label{pole}
\end{figure}

Pre tento špeciálny prípad veterného poľa sa systém diferenciálnych rovíc zjednoduší na:
\begin{subequations}
 \label{eq:55}
 \begin{align}
  \label{eq:56}
  \dd{x_{\navext}}{t} &= V \cos \beta_{\navext} - \frac{V}{h}y,  \\
  \label{57}
  \dd{y_{\navext}}{t} &= V \sin \beta_{\navext},  \\
  \label{58}
  \dd{\beta_\navext}{t}
                              &=
  \frac{V}{h}  \cos^2 \beta_\navext.                         
 \end{align}

\end{subequations}
Poslednú rovnicu vieme vyriešiť separáciou premenných:
\begin{align}
 \label{eq:59}
\tan \beta_\navext - \tan \beta_{\navext, B} &= \frac{V}{h}(t-t_{B}),
\end{align}
kde sme použili značenie  $ \left. \beta_{\navext, B}=_{\bydefinition} \beta_{\navext} \right|_{t=t_{\navend, \navext}}   $.  Nakoľko je funkcia $\beta_\navext$ rastúca funkcia času, môžeme zmeniť premenné a prepísať rovnicu \ref{57} ako $\dd{y_{\navext}}{\beta_\navext}\dd{\beta_\navext}{t} = V \sin \beta_\navext$, z čoho dostávame:
\begin{align}
\dd{y_\navext}{\beta_\navext}=h\frac{\sin \beta_\navext}{\cos ^2 \beta_\navext}.
\end{align}
Z toho jednoducho:
\begin{align}
 \label{eq:60}
y_\navext(\beta_\navext)=h\left(\frac{1}{\cos \beta_\navext}-\frac{1}{\cos \beta_{\navext,B}}\right).
\end{align}
Potrebujeme $y_\navext(\beta_{\navext,B})=0$. Nakoniec môžeme uskutočniť rovnakú zmenu premenných v \ref{eq:56}, čo dáva:
\begin{align}
\dd{x_\navext}{\beta_\navext}=h\left(\frac{1}{\cos \beta_\navext}-\frac{1}{\cos^3 \beta_\navext}+\frac{1}{\cos^2 \beta_\navext \cos \beta_{\navext,B}}\right).
\end{align} 
Riešenie v tvare:
\begin{align}
 \label{eq:61}
x_\navext(\beta_\navext)=\frac{1}{2}h(-\arctanh \sin \beta_{\navext,B} + \arctanh \sin \beta_\navext - \sec \beta_{\navext,B} \tan \beta_{\navext,B} + 2\sec \beta_{\navext,B} \tan \beta_\navext - \sec \beta_\navext \tan \beta_\navext),
\end{align}
potrebujeme aby platilo $x_\navext(\beta_{\navext,end})=0$. Teraz môžeme použiť rovnice \ref{eq:61} a \ref{eq:60}. Počiatočné podmienky musia vyhovovať \ref{eq:2}, teda dostávame systém rovníc:
\begin{align}
 \label{eq:62}
\vec{x}(t_{A})
=
\begin{bmatrix}
    \frac{1}{2}h(-\arctanh \sin 	       \beta_{\navext,B} + \arctanh \sin \beta_\navext - \frac{1}{\cos \beta_{\navext,B}} \tan \beta_{\navext,B} + \frac{2}{\cos \beta_{\navext,B}} \tan \beta_\navext - \frac{1}{\cos \beta_\navext} \tan \beta_\navext) \\
    h\left(\frac{1}{\cos \beta_\navext}-\frac{1}{\cos \beta_{\navext,B}}\right)
  \end{bmatrix}.
\end{align}
Nakoľko poznáme $\vec{x}(t_{A})$, dostávame teda z \ref{eq:62} sústavu dvoch nelineárnych algebraických rovníc pre dve neznáme $\beta_{\navext,end}$ a $\beta_\navext$. Akonáhle nájdeme tieto dve hodnoty, môžeme z rovnice \ref{eq:59} vyjadriť konečný čas.

Konečne môžeme konštatovať, že pre výpočet úlohy navigácie lode, potrebujeme pre dané $V$, $h$ a 
$\vec{x}(t_{A})= \transpose {\begin{bmatrix}
    x(t_{A}) &
    y(t_{A})
  \end{bmatrix}
},
$
vyriešiť systém nelineárnych algebraických rovníc:
\begin{align}
 \label{eq:63}
\begin{bmatrix}
x(t_{A})\\
y(t_{A})
\end{bmatrix}
=
\begin{bmatrix}
    \frac{1}{2}h(-\arctanh \sin 	       \beta_{\navext,B} + \arctanh \sin \beta_\navext - \frac{1}{\cos \beta_{\navext,B}} \tan \beta_{\navext,B} + \frac{2}{\cos \beta_{\navext,B}} \tan \beta_\navext - \frac{1}{\cos \beta_\navext} \tan \beta_\navext) \\
    h\left(\frac{1}{\cos \beta_\navext}-\frac{1}{\cos \beta_{\navext,B}}\right)
  \end{bmatrix}
\end{align}
a dostaneme hodnoty $\beta_\navext$ a $\beta_{\navext,B}$. Z rovnice 
\begin{align}
\tan \beta_{\navext,A} - \tan \beta_{\navext, B} &= \frac{V}{h}(t_{A}-t_{B}),
\end{align}
nájdeme konečný čas $t_{B}$. Optimálna trajektória je daná ako riešenie systému diferenciálnych rovníc prvého stupňa:
\begin{subequations}
 \label{eq:64}
 \begin{align}
  \label{eq:65}
  \dd{x_{\navext}}{t} &= V \cos \beta_{\navext} - \frac{V}{h}y_\navext,  \\
  \label{eq:66}
  \dd{y_{\navext}}{t} &= V \sin \beta_{\navext},  \\
  \label{eq:67}
  \dd{\beta_\navext}{t}
                              &=
  \frac{V}{h}  \cos^2 \beta_\navext, 
 \end{align}
\end{subequations}
ktoré sa riešia v časovom intervale $t \in (t_{A}, t_{B})$, s ohľadom na počiatočné podmienky:
\begin{subequations}
 \label{eq:68}
 \begin{align}
  \label{eq:69}
  \left. x_\navext \right|_{t=t_{A}} &= x_{A},  \\
  \label{eq:70}
  \left. y_\navext \right|_{t=t_{A}} &= y_{A},   \\
  \label{eq:71}
  \left. \beta_\navext \right|_{t=t_{A}} &= \beta_{\navext, A}. 
 \end{align}
\end{subequations}
Problém \ref{eq:63} až \ref{eq:68} sa dá vyriešiť numerickými metódami.

\section{Numerické řešení}
\label{sec:NumJVP}
Jak jsme už naznačili, naše jednoduché větrné pole se dá vyřešit numericky\footnote{Veškeré výpočty jsou provedeny v programu Wolfram Mathematica, verze 12.2.}. Rozhodli jsme se pracovat s následujícími hodnotami\footnote{Naše hodnoty mají následující jednotky: $V$ a $"h1-h3"$ - \si{km.h^{-1}}, $x1$ a $x2$ - \si{km}.}:
\begin{lstlisting}[language=Mathematica, caption=Hodnoty]
V = 10;
h1 = 1;
h2 = 10;
h3 = 0.1;
x1 = "in. condition X";
x2 = "in. condition Y"
\end{lstlisting}

Pro rychlost vzducholodě $V$ jsme si brali jen jednu hodnotu. Při našich výpočtech se ukázalo, že různé hodnoty $V$ nemají vliv na trajektorii, takže ani na startovní úhel $\beta_{ext,A}$. Ovlivněna je jen doba letu.

Nazveme x-ovou část vektoru \eqref{eq:63} $X1$ a y-ovou část stejného vektoru $X2$. Za $x(t_{A})$ dáme $x1$ a za $y(t_{A})$ dáme $x2$. Získáme dvě rovnice: $X1=x1$ a $X2=x2$. V těchto rovnicích se nachází hodnota $h$, za kterou postupně dosazujeme hodnoty $"h1-h3"$. Pojďme vyřešit tuto soustavu nelineárních algebraických rovnic.

Pro numerické řešení použijeme metodu $FindRoot$. 
 
\begin{lstlisting}[language=Mathematica, caption=Metoda řešení,mathescape]
res = FindRoot[{X1 == x1, X2 == x2}, {{B1, "0-2$\pi$"}, {B2, "0-2$\pi$"}}]
\end{lstlisting}

Metoda \texttt{FindRoot} byla jediná, která v našich výpočtech fungovala. Bohužel je v tomto případě ne až tak praktická, kvůli dlouhému hledání nejvhodnějších parametrů $B1$ a $B2$.

Řešení \eqref{eq:63} můžeme pak použijeme k výpočtu času, za který se dostane vzducholoď z bodu A do bodu B.

\begin{lstlisting}[language=Mathematica, caption=Výpočet času,mathescape]
$\beta_{ext,A}$ = B1 /.res;
$\beta_{ext,B}$ = B2 /.res;
time = -(h/V)(Tan[$\beta_{ext,A}$] - Tan[$\beta_{ext,B}$])
\end{lstlisting}

Nakonec si vykreslíme trajektorie.
\begin{lstlisting}[language=Mathematica, caption=Vykreslení trajektorie,mathescape]
rce = {x'[t] == V*Cos[$\beta$[t]] - (V/"h1-h3")y[t];
y'[t] == V*Sin[$\beta$[t]];
$\beta$'[t] == (V/"h1-h3")(Cos[$\beta$[t]])^2};

sol = NDSolve[Join[rce, {x[0] == x1, y[0] == x2, $\beta$[0] == $\beta_{ext,A}$], {x, y, $\beta$},{t, 0, 3600}];

ParametricPlot[Table[{x[t], y[t]} /. sol], {t, 0, time}, AxesLabel -> {x, y}, PlotRange -> All]
\end{lstlisting}


\subsection{Osa X}
\label{sec: OsaX}

\begin{wrapfigure}{r}{.55\textwidth}
    \begin{minipage}{\linewidth}
    \centering
    \includegraphics[width=10cm]{figures/OsaX2.pdf}
    \caption{Trajektorie, start osa $x$, $h$ = 1}
    \label{Xtraj1}
    \includegraphics[width=10cm]{figures/OsaX3.pdf}
    \caption{Trajektorie, start osa $x$, $h$ = 10}
    \label{Xtraj10}
    \includegraphics[width=10cm]{figures/OsaX4.pdf}
    \caption{Trajektorie, start osa $x$, $h$ = 0.1}
    \label{Xtraj0.1}
    \includegraphics[width=10cm]{figures/OsaX1.pdf}
    \caption{Trajektorie, start osa $x$}
    \label{Xtraj}
\end{minipage}
\end{wrapfigure}

Začneme s případem - start z osy $x$.

Startujeme tak, že vplujeme do větrného pole z bezvětří a pomalu/rychle se necháme unášet k našemu cíli. Tento případ je nejintuitivnější. Zhruba si dokážeme představit trajektorii, podél které se vzducholoď pohybuje.

Pro lepší představu se můžeme podívat na obrázky \eqref{Xtraj1}, \eqref{Xtraj10} a \eqref{Xtraj0.1}, kde vykreslujeme trajektorie pro různé hodnoty $h$. Pro porovnání těchto trajektorií se můžeme kouknout na obrázek \eqref{Xtraj}. V tabulce \eqref{TabX1} máme délku doby letu v hodinách a v tabulce \eqref{TabX2} máme startovní úhel v radiánech.

\subsection{Osa Y}
\label{sec: OsaY}

Koukneme se na start z osy $y$.

Tento případ začíná být už zajímavější. Pro velká $h$ a malá $y$ je trajektorie podobná trajektorii, která začíná z osy $x$. Na obrázku \eqref{Ytraj} lze pozorovat, že čím je $y$ dále od počátku, tak se trajektorie protahuje. Toto samé platí, když hodnotu $h$ snižujeme, což lze pozorovat na obrázku \eqref{Ytraj0.1}. V tabulkách \eqref{TabY1} a \eqref{TabY2} máme časy a startovní úhly pro různé pozice. Když porovnáme časy, které jsem získali při studování startů z osy $x$ a osy $y$, můžeme si všimnout, že jsme zpravidla rychleji v cíli, když startujeme z osy $x$. 
\subsection{Kvadranty}
\label{sec: Kvad}
Zamíříme k obecnějšímu případu - start z kvadrantů.

Vybrali jsme jen tři různé počáteční podmínky ze dvou kvadrantů (1. a 2.). Proč jen ze dvou kvadrantů? Jak jsme mohli vypozorovat ze startů z os $x$ a $y$, tak trajektorie jsou středově souměrná podle počátku (pokud je i jejich počáteční startovní bod středově souměrný podle počátku)\footnote{Tato vlastnost plyne ze symetrie větrného pole - je také středově souměrné podle počátku.}, takže není nutné studovat chování v 3. a 4. kvadrantu.

Na obrázku \eqref{Ktraj1} můžeme pozorovat, jak se mění tvary trajektorií v závislosti na počátečních podmínkách. Na obrázku \eqref{Ktraj10} lze vidět, že když zvětšíme $h$, tak trajektorie budou mít tendenci mířit přímo do cíle. V tabulkách \eqref{TabK1} a \eqref{TabK2} uvádíme zase dobu letu a startovní úhel.


\begin{figure}[h]
  \centering
  \subfloat[Doba letu v hodinách]{
  \label{TabX1}
  \includegraphics[scale=0.7]{figures/OsaX - tab. čas.pdf}
  }
  \qquad
  \subfloat[Startovní úhel $\beta_{\navext}(t_{A})$  v radiánech]{
  \label{TabX2}
  \includegraphics[scale=0.7]{figures/OsaX - tab. beta.pdf}
  }
\caption{Osa $x$}
\end{figure}

\begin{figure}[ht]
  \centering
  \subfloat[$h$ = 1 (nečárkované), $h$ = 10 (čárkované)]{
  \label{Ytraj}
  \includegraphics[scale=0.65]{figures/OsaY1.pdf}
  }
  \qquad
  \subfloat[$h$ = 0.1]{
  \label{Ytraj0.1}
  \includegraphics[scale=0.65]{figures/OsaY4.pdf}
  }
  \qquad
  \subfloat[Doba letu v hodinách]{
  \label{TabY1}
  \includegraphics[scale=0.6]{figures/OsaY - tab. čas.pdf}
  }
  \qquad
  \subfloat[Startovní úhel $\beta_{\navext}(t_{A})$  v radiánech]{
  \label{TabY2}
  \includegraphics[scale=0.6]{figures/OsaY - tab. beta.pdf}
  }
\caption{Osa $y$}
\end{figure}

\begin{figure}[ht]
  \centering
  \subfloat[$h$ = 1]{
  \label{Ktraj1}
  \includegraphics[scale=0.65]{figures/Kvad2.pdf}
  }
  \qquad
  \subfloat[$h$ = 10]{
  \label{Ktraj10}
  \includegraphics[scale=0.65]{figures/Kvad3.pdf}
  }
  \qquad
  \subfloat[Trajektorie, startů v kvadrantech]{
  \label{Ktraj}
  \includegraphics[scale=0.9]{figures/Kvad1.pdf}
  }
  \qquad
  \subfloat[Doba letu v hodinách]{
  \label{TabK1}
  \includegraphics[scale=0.6]{figures/Kvad - tab. čas.pdf}
  }
  \qquad
  \subfloat[Startovní úhel $\beta_{\navext}(t_{A})$  v radiánech]{
  \label{TabK2}
  \includegraphics[scale=0.6]{figures/Kvad - tab. beta.pdf}
  }
\caption{Kvadranty}
\end{figure}


%\begin{figure}
%\includegraphics[scale=0.7]{figures/Kvad2.pdf}
%\caption{Trajektorie, start v kvadrantu, $h$ = 1}
%\label{Ktraj1}
%\end{figure}

%\begin{figure}
%\includegraphics[scale=0.7]{figures/Kvad3.pdf}
%\caption{Trajektorie, start v kvadrantu, $h$ = 10}
%\label{Ktraj10}
%\end{figure}

%\begin{figure}
%\includegraphics[scale=0.7]{figures/Kvad1.pdf}
%\caption{Trajektorie, start v kvadrantu}
%\label{Ktraj}
%\end{figure}

%\begin{figure}
%\includegraphics[scale=0.7]{figures/Kvad - tab. čas.pdf}
%\caption{Tabulka - doba letu (hodiny)}
%\label{TabK1}
%\end{figure}
%
%\begin{figure}
%\includegraphics[scale=0.7]{figures/Kvad - tab. beta.pdf}
%\caption{Tabulka - startovní úhel (radiány)}
%\label{TabK2}
%\end{figure}




















%\begin{lstlisting}[language=Mathematica, caption=Konstanty]
%g = 9.81;
%l = 1;
%poc = 1;
%time = {t, 0, 10};
%\end{lstlisting}




\bibliographystyle{custom}
\bibliography{ref}

\addtocontents{toc}{\protect\end{multicols}} % workaround for table of contents in two columns in amsart documentclass
\end{document}

%%% Local Variables: 
%%% mode: latex
%%% TeX-master: t
%%% End: 
