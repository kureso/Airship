\documentclass[reqno, a4paper]{amsart}
\newcommand\hmmax{0}
\newcommand\bmmax{0}
\usepackage{amsmath}
\usepackage{amssymb}
\usepackage{amsthm}

% PAGE DIMENSION
\usepackage[scale=0.9]{geometry}

% BIBLIOGRAPHY
\usepackage[authoryear]{natbib}
\usepackage{bibentry} % inline refereces

% ENCODING, LANGUAGE
\usepackage[czech]{babel}
\usepackage[utf8]{inputenc}

% GRAPHICS
\usepackage{subfig}
\usepackage{graphicx}

% HYPERTEXT, SOURCE CODE SPECIALS
\usepackage[unicode]{hyperref}
\usepackage[active]{srcltx} % use TeX-souce-specials-mode

% SYMBOLS, FONTS
\usepackage{mathbbol}
\usepackage{bm} % sophisticated \boldsymbol
%\usepackage{stmaryrd}
\usepackage{MnSymbol} % \lsem, \rsem, tensor product :
\usepackage{gensymb}
\usepackage{eurosym}

% UNITS, TYPESETTING TENSORS
%\usepackage{units}
\usepackage{siunitx}
\usepackage{tensor}
\usepackage{accents}

% COMPACT LIST ENVIRONMENT
\usepackage{enumitem}

% LINE NUMBERS
\usepackage{lineno}

% TABLE OF CONTENTS IN TWO COLUMNS
% \usepackage[toc]{multitoc} % It seems that it does not work with amsart
% the workaround is the command
% \addtocontents{toc}{\protect\begin{multicols}{2}} % workaround for table of contents in two columns in amsart documentclass
% see below
\usepackage{multicol}

% TODO NOTES
\usepackage{todonotes}

% TABLES
\usepackage{booktabs}
\usepackage{dcolumn}
\usepackage{tabularx}

% SOURCE CODE LISTINGS
\usepackage{listings}
%\usepackage{minted}
\usepackage{xcolor}

% IMPORT CSV files
\usepackage{csvsimple}

% NUMBERS
%\usepackage{numprint}

\numberwithin{equation}{section}
\let\cite\citet
\newcommand*{\doi}[1]{\href{http://dx.doi.org/#1}{doi: #1}}

\input{macros}
\input{listings-style}
\newcommand{\navstart}{A} % subscript denoting start of the navigation (time, initial positions, ...)
\newcommand{\navend}{B} % subscript denoting the end of the navigation (time, initial positions, ...)

\newcommand{\navext}{\mathrm{ext}} % extreme value

\title[Matematické kyvadlo]{Navigace vzducholodi}

\author{Ondrej Kureš, Marek Mikloš, Ladislav Trnka}

%\thanks{}


\begin{document}

\begin{abstract}
Řešíme problém navržený \citep{prusa.tuma}
\end{abstract}

\maketitle

\addtocontents{toc}{\protect\begin{multicols}{2}} % workaround for table of contents in two columns in amsart documentclass
\tableofcontents

%\linenumbers


\section{Úvod}
\label{sec:Úvod}

Vzducholoď se pohybuje ve větrném poli $\vec{w}$ a má za cíl překonat vzdálenost z bodu A do bodu B. V tomto textu se budeme zabývat otázkou jak zvolit její trasu, aby dorazila do cíle v nejkratším možném čase. Točení kormidla vzducholodi budeme charakterizovat jejím směrem letu tedy funkcí $\beta (t)$. Můžeme se ptát, jak točit kormidlem tak, aby vzducholoď dorazila do cíle co nejdříve. 

Trajektorii vzducholodi budeme popisovat v kartézských souřadnicích a to v rovině $(x,y)$, zanedbáme popis výšky. Vzducholoď se v bezvětří pohybuje rychlostí $\vec{V}$. Pro zjednodušení výpočtů uvažujme konstantní rychlost $\vec{V}$, stacionární pole $\vec{w}$ a cílový bod B jako počátek souřadnic (lze vždy zajistit vhodnou transformací). 

\begin{figure}[h!]
  \centering
  \includegraphics[width=15cm]{figures/airship.eps}
  \caption{Nastínění uvažované situace.}
  \label{Rplot1-2}
\end{figure}

Pro okamžitou rychlost vzducholodi platí:
\begin{equation} 
\label{rce1}
\begin{split}
\dd{x}{t} & = V \cos \beta(t) + u(x,y), \\
\dd{y}{t} & = V \sin \beta(t) + v(x,y),
\end{split}
\end{equation}

kde $
\vec{x}(t)=\transpose{
  \begin{bmatrix}
    x(t) &
    y(t)
  \end{bmatrix}
}
$ je hledaná trajektorie, $\beta \in \langle 0, 2 \pi)$ je směr letu a $
\vec{w}
=
\transpose{
  \begin{bmatrix}
    u &
    v
  \end{bmatrix}
}
$ je dané pole větru. Dále známe:
\begin{subequations}
  \label{eq:1}
  \begin{align}
    \label{eq:2}
    \vec{x}(t_{A})& = A, \\
    \label{eq:3}
    \vec{x}(t_{B})& = B = \transpose{
  \begin{bmatrix}
    0 &
    0
  \end{bmatrix}
},
  \end{align}
\end{subequations}
kde $t_{A}$ je čas startu vzducholodi a $t_{B}$ je čas příletu\footnote{Při příletu vzducholoď nebude mít nulovou rychlost.}. 

\section{Variační počet}
\label{sec:Variační počet}

Náš zájem se proto soustřeďuje na minimalizaci funkcionálu:
\begin{equation}
  \label{eq:8}
  I(\beta, t_{B}) =_{\bydefinition} \int_{t_{A}}^{t_{B}}\, \diff t = t_{B} - t_{A},
\end{equation}
při splnění soustavy rovnic $\eqref{rce1}$, které kompaktněji přepišme jako:
\begin{equation}
  \label{eq:7}
  \dd{\vec{x}}{t} = \vec{f}(\vec{x}, \beta).
\end{equation}
Chceme tedy minimalizovat cestovní čas a přípustné trajektorie musí splňovat $\eqref{eq:7}$. Při hledání extremály využijme koncept vázaných extrémů a Lagrangeových multiplikátorů $\vec{\lambda}$. Proto studujme funkcionál:
\begin{equation}
  \label{eq:9}
  J(\beta, t_{B}) =_{\bydefinition}
  \int_{t = t_{A}}^{t_{B}}
  \left(
    1
    -
    \vectordot{\greekvec{\lambda}}{\left(\dd{\vec{x}}{t} - \vec{f}(\vec{x}, \beta) \right)}
  \right)
  \, \diff t,
\end{equation}
kde funkce $\vec{\lambda}$ bude upřesněna později. Nyní hledejme G\^{a}teauxovu derivaci $J(\beta, t_{B})$:
\begin{equation}
\begin{split}
  \label{eq:13}
  \Diff J(\beta, t_{B}) [(\alpha, \tau)]
  & =
  _{\bydefinition}
  \left.
    \dd{}{\varepsilon}
    J(\beta_{\navext} + \varepsilon \alpha, t_{\navend, \navext} + \varepsilon \tau)
  \right|_{\varepsilon = 0}  \\
  & =
  _{\bydefinition}
  \left.
    \left[
      \dd{}{\varepsilon}
      \int_{t = t_{A}}^{t_{\navend, \navext} + \varepsilon \tau}
      \left(
        1
        -
        \vectordot{\greekvec{\lambda}}{\left(\dd{\vec{x}_{\varepsilon}}{t} - \vec{f}(\vec{x}_{\varepsilon}, \beta) \right)}
      \right)
      \, \diff t
    \right]
  \right|_{\varepsilon = 0}
  ,
\end{split}
\end{equation}
při variaci:
\begin{subequations}
  \label{eq:10}
  \begin{align}
    \label{eq:11}
    \beta &= \beta_{\navext} + \varepsilon \alpha, \\
    \label{eq:12}
    t_{\navend} &= t_{\navend, \navext} + \varepsilon \tau,
  \end{align}
\end{subequations}
kde $\vec{x}_{\epsilon}$ je korespondující trajektorie k $\beta$ a $t_{B}$. Přičemž stále platí:
\begin{subequations}
  \label{eq:010}
  \begin{align}
    \label{eq:020}
    \vec{x}_{\epsilon}(t_{A})& = \vec{x}_{\navext}(t_{A})=A, \\
    \label{eq:030}
    \vec{x}_{\epsilon}(t_{\navend, \navext} + \varepsilon \tau) & =\vec{x}_{\navext}(t_{\navend, \navext})= B = \vec{0}.
  \end{align}
\end{subequations}
Nejdříve upravme $\eqref{eq:13}$ pomocí integrace per partes na člen $\vectordot{\greekvec{\lambda}}{\dd{\vec{x}_{\varepsilon}}{t}}$, některé členy budou dle předchozího nulové a dostáváme:
\begin{multline}
  \label{eq:18}
  \Diff J(\beta, t_{B}) [(\alpha, \tau)]
  =
  \left.
    \left[
      \dd{}{\varepsilon}
      \int_{t = t_{\navstart}}^{t_{\navend, \navext}}
      \left(
        1
        +
        \vectordot{\dd{\greekvec{\lambda}}{t}}{\vec{x}_{\varepsilon}}
        +
        \vectordot{\greekvec{\lambda}}{\vec{f}(\vec{x}_{\varepsilon}, \beta)}
      \right)
      \, \diff t
    \right]
  \right|_{\varepsilon = 0}
  +
  \left.
    \left[
      \dd{}{\varepsilon}
      \int_{t = t_{\navend, \navext}}^{t_{\navend, \navext} + \varepsilon \tau}
      \left(
        1
        +
        \vectordot{\dd{\greekvec{\lambda}}{t}}{\vec{x}_{\varepsilon}}
        +
        \vectordot{\greekvec{\lambda}}{\vec{f}(\vec{x}_{\varepsilon}, \beta)}
      \right)
      \, \diff t
    \right]
  \right|_{\varepsilon = 0}
  \\
  =
    \left.
    \left[
      \dd{}{\varepsilon}
      \int_{t = t_{\navstart}}^{t_{\navend, \navext}}
      \left(
        1
        +
        \vectordot{\dd{\greekvec{\lambda}}{t}}{\vec{x}_{\varepsilon}}
        +
        \vectordot{\greekvec{\lambda}}{\vec{f}(\vec{x}_{\varepsilon}, \beta)}
      \right)
      \, \diff t
    \right]
  \right|_{\varepsilon = 0}
  +
\left.
    \left[
        1
        +
        \vectordot{\greekvec{\lambda}}{\vec{f}(\vec{x}_{\navext}, \beta_{\navext})}
    \right]
  \right|_{t = t_{\navend, \navext}}
  \tau.
\end{multline}
Podle \citep{prusa.tuma} použijeme geniální trik:
$\vec{x}_{\varepsilon} \approx \vec{x}_{\navext} + \epsilon \vec{y} + \cdots$, kde zanedbáme členy vyššího řádu a kde $\vec{y}$ je funkce času. Tím dále můžeme upravit první člen v poslední rovnosti $\eqref{eq:18}$:
\begin{align}
    \left.
    \left[
      \dd{}{\varepsilon}
      \int_{t = t_{\navstart}}^{t_{\navend, \navext}}
      \left(
        1
        +
        \vectordot{\dd{\greekvec{\lambda}}{t}}{\vec{x}_{\varepsilon}}
        +
        \vectordot{\greekvec{\lambda}}{\vec{f}(\vec{x}_{\varepsilon}, \beta)}
      \right)
      \, \diff t
    \right]
  \right|_{\varepsilon = 0} =  \int_{t = t_{\navstart}}^{t_{\navend, \navext}}
  \left(
    \vectordot{\left[ \dd{\greekvec{\lambda}}{t} + \transpose{\left. \pd{\vec{f}}{\vec{x}} \right|_{\vec{x} = \vec{x}_{\navext}, \beta = \beta_{\navext}}} \greekvec{\lambda} \right]}{\vec{y}}
    +
    \vectordot{\greekvec{\lambda}}
    {
      \left. \pd{\vec{f}}{\beta} \right|_{\vec{x} = \vec{x}_{\navext}, \beta = \beta_{\navext}} \alpha
    }
  \right)
  \, \diff t.
\end{align}
Nyní můžeme přistoupit k vybraní $\vec{\lambda}$ takové, aby bylo splněno:
\begin{align}
  \label{eq:24}
  \dd{\greekvec{\lambda}}{t} = - \transpose{\left. \pd{\vec{f}}{\vec{x}} \right|_{\vec{x} = \vec{x}_{\navext}, \beta = \beta_{\navext}}} \greekvec{\lambda}
  .
\end{align}
Po dosazení dostáváme výsledný vztah pro G\^{a}teuxovu derivaci:
\begin{equation}
  \label{eq:25}
  \Diff J(\beta, t_{B}) [(\alpha, \tau)]
  =
  \int_{t = t_{\navstart}}^{t_{\navend, \navext}}
  \vectordot{\greekvec{\lambda}}
  {
    \left. \pd{\vec{f}}{\beta} \right|_{\vec{x} = \vec{x}_{\navext}, \beta = \beta_{\navext}} \alpha
  }
  \, \diff t
  +
  \left.
    \left[
        1
        +
        \vectordot{\greekvec{\lambda}}{\vec{f}(\vec{x}_{\navext}, \beta_{\navext})}
    \right]
  \right|_{t = t_{\navend, \navext}}
  \tau =0
  ,
\end{equation}
což musí platit pro libovolně $\alpha$ a $\tau$. Tímto dostáváme:
\begin{subequations}
  \label{eq:rce2}
  \begin{align}
    \label{eq:31}
    \vectordot{\greekvec{\lambda}}
    {
    \pd{\vec{f}}{\beta}(\vec{x}_{\navext}, \beta_{\navext}) 
    }
    &=
      0, \\
    \label{eq:32}
    \left.
    \left[
    1
    +
    \vectordot{\greekvec{\lambda}}{\vec{f}(\vec{x}_{\navext}, \beta_{\navext})}
    \right]
    \right|_{t = t_{\navend, \navext}}
    &=
      0,
  \end{align}
\end{subequations} 

....

\section{Závěr}
\label{sec:Závěr}
Lorem ipsum dolor sit amet, consectetuer adipiscing elit. Nullam sit amet magna in magna gravida vehicula. Nullam eget nisl. In rutrum. Itaque earum rerum hic tenetur a sapiente delectus, ut aut reiciendis voluptatibus maiores alias consequatur aut perferendis doloribus asperiores repellat. Maecenas sollicitudin. Integer malesuada.\citep{prusa.tuma}




























%\begin{lstlisting}[language=Mathematica, caption=Konstanty]
%g = 9.81;
%l = 1;
%poc = 1;
%time = {t, 0, 10};
%\end{lstlisting}




\bibliographystyle{custom}
\bibliography{ref}

\addtocontents{toc}{\protect\end{multicols}} % workaround for table of contents in two columns in amsart documentclass
\end{document}

%%% Local Variables: 
%%% mode: latex
%%% TeX-master: t
%%% End: 
